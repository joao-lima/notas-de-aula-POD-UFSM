%%%%%%%%%%%%%%%%%%%%%%%%%%%%%%%%%%%%%%%%%%%%%%%%%%%%%%%%%%%%%%%%%%%%%%%%%%%%%%%
\section{Aula 02}
%%%%%%%%%%%%%%%%%%%%%%%%%%%%%%%%%%%%%%%%%%%%%%%%%%%%%%%%%%%%%%%%%%%%%%%%%%%%%%%

%%%%%%%%%%%%%%%%%%%%%%%%%%%%%%%%%%%%%%%%%%%%%%%%%%%%%%%%%%%%%%%%%%%%%%%%%%%%%%%
\subsection{Definição de algoritmo}

\begin{itemize}
\item Procedimento bem definido que:
	\begin{itemize}
	\item recebe como entrada um valor, ou conjunto de valores
	\item produz como saída um valor ou conjunto de valores
	\end{itemize}
\item Conjunto de passos que transforma a entrada na saída.
\item Exemplo do {\bf problema de ordenação}:
\end{itemize}

{\bf Entrada}: uma sequência de $n$ números $\langle a_1, a_2, ..., a_n \rangle$.

{\bf Saída}:  Uma permutação $\langle {a'}_1, {a'}_2, ..., {a'}_n \rangle$ da
entrada tal que ${a'}_1 \leq {a'}_2 \leq ... \leq {a'}_n \rangle$.

\begin{itemize}
\item Essa entrada do algoritmo é uma {\bf instância} do problema.
\item Para um algoritmo estar {\bf correto}, para cada instância, ele termina com saída 
correta. Um algoritmo correto \textsf{resolve} um problema. 
\end{itemize}

Exemplos de problemas:
\begin{itemize}
\item O Projeto do Genoma Humano para identificar todos os genes do DNA humano.
\item Internet com descoberta de rotas.
\item E-commerce.
\end{itemize}

%%%%%%%%%%%%%%%%%%%%%%%%%%%%%%%%%%%%%%%%%%%%%%%%%%%%%%%%%%%%%%%%%%%%%%%%%%%%%%%
\subsection{Análise de algoritmos}

\subsubsection{Eficiência}

\begin{itemize}
\item Medida de custo de execução de um programa.
\item Importante separar entre uma medida {\bf real}  e {\bf generalizada}.
	\begin{itemize}
	\item real depende do compilador, hardware, memória, etc.
	\item generalizada pode usar um computador abstrato para análise.
		\begin{itemize}
		\item Ex: máquina teóricas (Turing, lambda).
		\end{itemize}
	\end{itemize}
\end{itemize}

\begin{framed}
\centering
Um algoritmo é eficiente quando seu tempo de execução é polinomial.
\end{framed}

Vamos considerar dois exemplos de algoritmos de ordenação com $n$ elementos de entrada:
\begin{itemize}
\item {\bf Inserção} com custo $c_1 n^2$ sendo $c_1$ uma constante que não depende de $n$.
\item {\bf Merge sort} com custo $c_2 n \log n$. 
\end{itemize}


\subsubsection{Ordem de crescimento}

\begin{table}[ht]
\centering
\begin{tabular}{cc}
\hline
{\bf Descrição} & {\bf Ordem} \\ 
\hline
constante        & $1$ \\
logaritmo        & $\log n$ \\
linear           & $n$ \\
linear-logaritmo & $n \log n$ \\
quadratico       & $n^2$ \\
cúbico           & $n^3$ \\
exponencial      & $2^n$ \\
\hline
\end{tabular}
\end{table}

