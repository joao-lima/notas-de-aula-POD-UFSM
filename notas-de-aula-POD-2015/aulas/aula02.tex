%%%%%%%%%%%%%%%%%%%%%%%%%%%%%%%%%%%%%%%%%%%%%%%%%%%%%%%%%%%%%%%%%%%%%%%%%%%%%%%
\section{Aula 02}
%%%%%%%%%%%%%%%%%%%%%%%%%%%%%%%%%%%%%%%%%%%%%%%%%%%%%%%%%%%%%%%%%%%%%%%%%%%%%%%

%%%%%%%%%%%%%%%%%%%%%%%%%%%%%%%%%%%%%%%%%%%%%%%%%%%%%%%%%%%%%%%%%%%%%%%%%%%%%%%
\subsection{Definição de algoritmo}

\begin{itemize}
\item Procedimento bem definido que:
	\begin{itemize}
	\item recebe como entrada um valor, ou conjunto de valores
	\item produz como saída um valor ou conjunto de valores
	\end{itemize}
\item Conjunto de passos que transforma a entrada na saída.
\item Exemplo do {\bf problema de ordenação}:
\end{itemize}

{\bf Entrada}: uma sequência de $n$ números $\langle a_1, a_2, ..., a_n \rangle$.

{\bf Saída}:  Uma permutação $\langle {a'}_1, {a'}_2, ..., {a'}_n \rangle$ da
entrada tal que ${a'}_1 \leq {a'}_2 \leq ... \leq {a'}_n \rangle$.

\begin{itemize}
\item Essa entrada do algoritmo é uma {\bf instância} do problema.
\item Para um algoritmo estar {\bf correto}, para cada instância, ele termina com saída 
correta. Um algoritmo correto \textsf{resolve} um problema. 
\end{itemize}

Exemplos de problemas:
\begin{itemize}
\item O Projeto do Genoma Humano para identificar todos os genes do DNA humano.
\item Internet com descoberta de rotas.
\item E-commerce.
\end{itemize}

%%%%%%%%%%%%%%%%%%%%%%%%%%%%%%%%%%%%%%%%%%%%%%%%%%%%%%%%%%%%%%%%%%%%%%%%%%%%%%%
\subsection{Análise de algoritmos}

Pergunta fund
