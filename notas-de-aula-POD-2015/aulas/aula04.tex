%%%%%%%%%%%%%%%%%%%%%%%%%%%%%%%%%%%%%%%%%%%%%%%%%%%%%%%%%%%%%%%%%%%%%%%%%%%%%%%
\chapter{Classificação em memória secundária}
%%%%%%%%%%%%%%%%%%%%%%%%%%%%%%%%%%%%%%%%%%%%%%%%%%%%%%%%%%%%%%%%%%%%%%%%%%%%%%%

Classificação em memória secundária, ou ordenação em memória externa, consiste
em ordenar arquivos de tamanho maior que a memória principal (interna)
disponível.
Nos algoritmos de ordenação externa deve-se reduzir o número de acessos ao disco,
resposável por grande parte do custo do algoritmo.

%Uma restrição que pode acontecer é o armazenamento de dados sequencial, como em
%fitas, discos e tambores magnéticos,.
%Ou também, quando somente um registro pode ser acessado em um dado momento.
%Esta última é uma restrições forte comparada ao acesso em vetores.

Os três principais fatores que diferenciam ordenação externa de interna são:
\begin{enumerate}
\item O custo para acessar um registro.
\item Restrições ao acesso físico, como em fitas magnéticas.
\item Tecnologia empregada.
\end{enumerate}

A maioria dos métodos de ordenação externa utiliza a seguinte estratégia:
\begin{enumerate}
\item Quebrar o arquivo em blocos do tamanho da memória interna disponível.
\item Ordenar cada bloco.
\item Intercalar os blocos ordenados, fazendo várias passadas.
\item A cada passada são criados blocos ordenados cada vez maiores, até que todo 
o arquivo esteja ordenado.
\end{enumerate}

Esses passos são agrupados em \textbf{passos}. 
O {\bf estágio de classificação} envolve os passos 1 e 2, enquanto que
o {\bf estágio de intercalação} consiste nos passos 3 e 4.
Os métodos para ordenação externa devem reduzir o número de passadas no arquivo.

Os conceitos usados neste capítulo são:
\begin{itemize}
\item {\bf Arquivo} - estrutura física de armazenamento. 
\item {\bf Partição} - estrutura lógica de armazenamento, que pode ser composto de
diversas partições.
\end{itemize}

%%%%%%%%%%%%%%%%%%%%%%%%%%%%%%%%%%%%%%%%%%%%%%%%%%%%%%%%%%%%%%%%%%%%%%%%%%%%%%%
\section{Classificação}
%%%%%%%%%%%%%%%%%%%%%%%%%%%%%%%%%%%%%%%%%%%%%%%%%%%%%%%%%%%%%%%%%%%%%%%%%%%%%%%

Considera-se que a memória principal tem capacidade para armazenar
$M$ registros do arquivo a classificar.

%%%%%%%%%%%%%%%%%%%%%%%%%%%%%%%%%%%%%%%%%%%%%%%%%%%%%%%%%%%%%%%%%%%%%%%%%%%%%%%
\subsection{Classificação interna}

A {\bf classificação interna} consiste na leitura de $M$  registros
para a memória, ordenação desses registros por qualquer outro algoritmo
de classificação interna e gravação desses registros classificados
em uma partição. 
Todas as partições ordenadas contêm $M$ registros, exceto (talvez) a última.

A {\bf classificação interna} envolve os seguintes passos:
\begin{enumerate}
\item leitura de $M$ registros para a memória.
\item ordenação desses registros por um algoritmo de ordenação interna.
	\begin{itemize}
	\item quicksort, mergesort, etc.
	\end{itemize}
\item gravação desses registros classificados.
	\begin{itemize}
	\item em uma partição nova de um arquivo existente (ou em novo arquivo).
	\end{itemize}
\end{enumerate}

A classificação interna possui a desvantagem de não explorar a ordem
parcial existente no arquivo de entrada.

%%%%%%%%%%%%%%%%%%%%%%%%%%%%%%%%%%%%%%%%%%%%%%%%%%%%%%%%%%%%%%%%%%%%%%%%%%%%%%%
\subsection{Seleção por substituição}

A {\bf seleção por substituição}  aproveita a possível classificação
parcial do arquivo de entrada. 
Em média, o tamanho das partições obtidas pelo processo de seleção com substituição
é de $2 * M$.

Os passos do algoritmo de seleção por substituição são:
\begin{enumerate}
\item Ler $M$ registros do arquivo para a memória.
\item Selecionar, na memória, o registro com menor chave.
\item Gravar na partição de saída o registro com menor chave.
\item Substituir (em memória) o registro gravado pelo próximo registro
do arquivo de entrada.
\item Caso a chave deste último seja menor do que a chave recém-gravada.
	\begin{enumerate}
	\item considerá-lo ``congelado'' e ignorá-lo no restante do processamento.
	\end{enumerate}
\item Caso existam em memória registros não ``congelados''
	\begin{enumerate}
	\item volta ao passo {\bf (2)}.
	\end{enumerate}
\item Caso contrário
	\begin{enumerate}
	\item fechar a partição de saída.
	\item ``descongelar'' os registros ``congelados''.
	\item abrir nova partição de saída.
	\item voltar ao passo {\bf (2)}.
	\end{enumerate}
\end{enumerate}

A seleção por substituição tem a desvantagem que no final da partição grande, 
parte do espaço em memória principal está ocupado com registros ``congelados''.

%%%%%%%%%%%%%%%%%%%%%%%%%%%%%%%%%%%%%%%%%%%%%%%%%%%%%%%%%%%%%%%%%%%%%%%%%%%%%%%
\subsection{Seleção natural}

A classificação por {\bf seleção natural} reserva um espaço na memória 
secundário (``\emph{o reservatório}''), que abriga os registros
``congelados'' em um processo de substituição.

A formação de uma partição se encerra quando o reservatório estiver cheio, ou
quando terminarem os registros de entrada.
Quando um reservatório encher, a memória é descarregada na partição atual
e os dados são copiados do repositório à memória.

A classificação por seleção natural usa como memória interna $M$ e um reservatório
que comporta $R$ registros.
Para $M = R$, o tamanho médio das partições é $M * e$ onde $e = 2,7182....$
(número de Euler).

Os passos do algoritmo de seleção natural são:
\begin{enumerate}
\item Ler $M$ registros do arquivo para a memória.
\item Selecionar, na memória, o registro com menor chave.
\item Gravar na partição de saída o registro com menor chave.
\item Substituir (em memória) o registro gravado pelo próximo registro
do arquivo de entrada.
\item Caso a chave deste último seja menor do que a chave recém-gravada.
	\begin{enumerate}
	\item gravá-lo no reservatório.
	\item substituir (em memória) o registro gravado no reservatório pelo
	próximo registro do arquivo de entrada.
	\end{enumerate}
\item Caso ainda exista espaço livre no reservatório.
	\begin{enumerate}
	\item volta ao passo {\bf (2)}.
	\end{enumerate}
\item Caso contrário
	\begin{enumerate}
	\item fechar a partição de saída.
	\item copiar os registros do reservatório para a memória.
	\item esvaziar o reservatório.
	\item abrir nova partição de saída.
	\item voltar ao passo {\bf (2)}.
	\end{enumerate}
\end{enumerate}

%%%%%%%%%%%%%%%%%%%%%%%%%%%%%%%%%%%%%%%%%%%%%%%%%%%%%%%%%%%%%%%%%%%%%%%%%%%%%%%
\subsection{Comparação de processos}

A classificação interna gera as menores partições, mas simplifica o estágio
de intercalação por usar partições de mesmo tamanho.
Os processos de seleção exigem intercalação mais elaborada, porém geram 
partições maiores, e reduzem o tempo de processamento.
A seleçõa natural gera as maiores partições, porém utiliza mais operações de 
entrada e saída. Além de usar memória externa adicional.

Os processos de seleção podem utilizar Heapsort para a ordenação interna.

%%%%%%%%%%%%%%%%%%%%%%%%%%%%%%%%%%%%%%%%%%%%%%%%%%%%%%%%%%%%%%%%%%%%%%%%%%%%%%%
\section{Intercalação}
%%%%%%%%%%%%%%%%%%%%%%%%%%%%%%%%%%%%%%%%%%%%%%%%%%%%%%%%%%%%%%%%%%%%%%%%%%%%%%%

O objetivo da intercalação é {\bf transformar um conjunto de partições classificadas em uma 
única partição classificada}.
Considera-se que a existência de $R$ partições geradas no particionamento.
O ideal seria intercalar todas as partições de uma só vez e obter o arquivo classificado.
Porém, todo sistema impõe um limite no número de arquivos abertos simultaneamente. 
Esse número pode ser menos que o número de partições geradas.

A intercalação exige uma série de passos, e em cada passo:
\begin{enumerate}
\item registros são lidos de um conjunto de partições dos arquivos de entrada.
\item e gravados em outro conjunto de partições de arquivos de saída.
\end{enumerate}

As operações de entrada e saída (E/S) tem um alto custo na ordenação externa.
Uma medida de eficiência do estágio de intercalação é dada pelo número de passos 
sobre os dados dado pela equação:
\begin{equation*}
N_{passos} = \frac{No. \; total \; de \; registros \; lidos}{No. \; total \;  de \; registros \; no \; arquivo \; classificado}
\end{equation*}
O menor número possível é $2$ ($1$ leitura e $1$ escrita).

%%%%%%%%%%%%%%%%%%%%%%%%%%%%%%%%%%%%%%%%%%%%%%%%%%%%%%%%%%%%%%%%%%%%%%%%%%%%%%%
\subsection{Intercalação balanceada de n caminhos}

A {\bf intercalação balanceada de n caminhos} usa $n$ arquivos para a entrada
de registros e outros $n$ arquivos para a saída de registros.  Sendo $F$ o
número máximo de arquivos que podem estar abertos ao mesmo tempo, $F = 2 n$.
Dessa forma, a intercalação balanceada de $n-$caminhos utiliza no máximo $F/2$
caminhos.

O particionamento inicial pode usar qualquer estratégia de particionamento.
{\bf Cada partição é escrita em um dos arquivos de entrada de forma intercalada}.

Os passos envolvidos são:
\begin{enumerate}
\item primeiro passo: determinar o número de arquivos ($F$) que o algoritmo irá
manipular, sendo a primeira metade ($F/2$) será usado para entrada enquanto
a outra metade ($F/2$) para saída.

\item passo dois: distribuir todas as partições, de forma {\bf intercalada},
nos arquivos de entrada.

\item {\bf passo de intercalação}: intercalar as primeiras partições,
gravando o resultado em um dos $F/2$ arquivos de saída, em ordem.

\item no final de cada fase, o conjunto de partições de saída
torna-se o conjunto de entrada.

\item a intercalação termina quando, em uma fase, grava-se apenas uma partição
(ordenada).
\end{enumerate}

%%%%%%%%%%%%%%%%%%%%%%%%%%%%%%%%%%%%%%%%%%%%%%%%%%%%%%%%%%%%%%%%%%%%%%%%%%%%%%%
\subsection{Intercalação ótima}

A {\bf intercalação ótima} usa $n$ arquivos para entrada e $1$ para saída
sendo $F = n + 1$.

O particionamento inicial pode usar qualquer estratégia de particionamento.
Ao contrário do particionamento na intercalação balanceada, {\bf o particionamento
inicial cria um novo arquivo por partição}.

Os passos envolvidos são:
\begin{enumerate}
\item primeiro passo: determinar o número de arquivos ($F$) que o algoritmo irá
manipular, sendo $F-1$ para entrada enquanto $1$ será de saída.

\item passo dois: abrir $F-1$ arquivos, cada um com uma partição. 

\item {\bf passo de intercalação}: intercalar as partições dos arquivos de entrada,
gravando o resultado em ordem no arquivo de saída.

\item o arquivo de saída é novo no primeiro passo, e é um dos arquivos já
inteiramente lidos nos próximos passos.

\item quando todas as partições ficarem vazias, os arquivos usados podem ser descartados.
O arquivo de saída vira um novo arquivo de entrada e é adicionado aos demais arquivos
de entrada não processados.

\item a intercalação termina quando restar apenas $1$ arquivo de entrada.
\end{enumerate}

O problema da intercalação ótima é que cada partição usa um arquivo próprio.
Esse número de arquivos pode ser grande.

%%%%%%%%%%%%%%%%%%%%%%%%%%%%%%%%%%%%%%%%%%%%%%%%%%%%%%%%%%%%%%%%%%%%%%%%%%%%%%%
\subsection{Intercalação polifásica}

A {\bf intercalação polifásica} permite restringir o número de arquivos usados.
Ela também usa $F-1$ arquivos de entrada, mas permite mútiplas partições por arquivo.

Os passos envolvidos são:
\begin{enumerate}
\item primeiro passo: determinar o número de arquivos ($F$) que o algoritmo irá
manipular, sendo $F-1$ para entrada enquanto $1$ será de saída.

\item passo dois: distribuir todas as partições, de forma {\bf intercalada},
nos arquivos de entrada.

\item {\bf passo de intercalação}: intercalar as partições dos arquivos de entrada,
gravando o resultado em ordem no arquivo de saída.

\item Se todas as partições selecionadas de entrada chegarem ao fim, 
mas nenhuma dos arquivos estiver no fim, uma nova particição no arquivo
de saída é iniciada.
As próximas partições de entrada são selecionadas.

\item Se um dos arquivos chegar ao fim, e todas as partições de entrada selecionadas
chegarem ao fim, o arquivo de saída se transforma em um arquivo de entrada.
O arquivo de entrada (que chegou ao fim) vira o arquivo de saída e uma nova fase é iniciada.

\item a intercalação termina quando restar apenas $1$ arquivo de entrada e 
$1$ partição.
\end{enumerate}

%%%%%%%%%%%%%%%%%%%%%%%%%%%%%%%%%%%%%%%%%%%%%%%%%%%%%%%%%%%%%%%%%%%%%%%%%%%%%%%
\subsection{Análise dos métodos de intercalação}

A tabela~\ref{aula04:tab:inter} mostra uma comparação dos métodos de intercalação
conforme exemplos.
%
\begin{table}[!ht]
\centering
\caption{Comparação dos métodos de intercalação.}
\begin{tabular}{l|c|c|c}
\hline
Estratégia & n. de operações & n. de passos & n. de arquivos usados \\ \hline
Intercalação balanceada & $72$ & $72/12 = 6$ & $4$ \\ \hline
Intercalação ótima      & $48$ & $48/12 = 4$ & $7$ \\ \hline
Intercalação polifásica & $60$ & $60/12 = 4$ & $4$ \\ \hline
%
\end{tabular}
\label{aula04:tab:inter}
\end{table}

A intercalação ótima pode ser usada quando {\bf não houver restrição} quanto ao
número de arquivos que podem ser criados, ou quando sabe-se que o número de
partições é menor que o limite.

A intercalação polifásica pode ser usada quando {\bf haver restrição} quanto ao
número de arquivos que podem ser criados.

%%%%%%%%%%%%%%%%%%%%%%%%%%%%%%%%%%%%%%%%%%%%%%%%%%%%%%%%%%%%%%%%%%%%%%%%%%%%%%%
\section{Exercícios}
%%%%%%%%%%%%%%%%%%%%%%%%%%%%%%%%%%%%%%%%%%%%%%%%%%%%%%%%%%%%%%%%%%%%%%%%%%%%%%%

\begin{enumerate}
\item Dados os números $10, 8, 7, 11, 9, 13, 16, 12, 15, 14$, mostre as partições
que são criadas por classificação para os métodos: classificação interna, seleção
por substituição, seleção natural.
Assuma $M = 3$ e $R = 3$.

\item Descreva os passos dos algoritmos de classificação para os métodos:
classificação interna, seleção por substituição, seleção natural.

\item Fazer a intercalação das seguintes partições:
$20, 24 - 5, 15 - 8, 9 - 12, 18 - 4, 10 - 7, 11 - 2, 13 - 14, 19$.
Utilize os métodos de: intercalação balanceada de n caminhos, intercalação ótima e intercalação
polifásica.
Consdere $F = 4$.

\end{enumerate}
