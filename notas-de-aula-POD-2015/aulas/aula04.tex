%%%%%%%%%%%%%%%%%%%%%%%%%%%%%%%%%%%%%%%%%%%%%%%%%%%%%%%%%%%%%%%%%%%%%%%%%%%%%%%
\chapter{Classificação em memória secundária}
%%%%%%%%%%%%%%%%%%%%%%%%%%%%%%%%%%%%%%%%%%%%%%%%%%%%%%%%%%%%%%%%%%%%%%%%%%%%%%%

Ou também, ordenação em memória externa, consiste em ordenar arquivos de tamanho
maior que a memória principal (interna) disponível.
Nos algoritmos de ordenação externa deve-se reduzir o número de acessos ao disco,
resposável por grande parte do custo do algoritmo.

Uma restrição que pode acontecer é o armazenamento de dados sequencial, como em
fitas, discos e tambores magnéticos,.
Ou também, quando somente um registro pode ser acessado em um dado momento.
Esta última é uma restrições forte comparada ao acesso em vetores.

Os três principais fatores que diferenciam ordenação externa de interna são:
\begin{enumerate}
\item O custo para acessar um registro.
\item Restrições ao acesso físico, como em fitas magnéticas.
\item Tecnologia empregada.
\end{enumerate}

A maioria dos métodos de ordenação externa utiliza a seguinte estratégia:
\begin{enumerate}
\item Quebrar o arquivo em blocos do tamanho da memória interna disponível.
\item Ordenar cada bloco.
\item Intercalar os blocos ordenados, fazendo várias passadas.
\item A cada passada são criados blocos ordenados cada vez maiores, até que todo 
o arquivo esteja ordenado.
\end{enumerate}
Esses passos são agrupados em \textbf{passos}. 
O {\bf estágio de classificação} envolve os passos 1 e 2, enquanto que
o {\bf estágio de intercalação} consiste nos passos 3 e 4.

Os métodos para ordenação externa devem reduzir o número de passadas no arquivo.

Os conceitos usados em seguida são:
\begin{itemize}
\item {\bf Arquivo} - estrutura física de armazenamento. 
\item {\bf Partição} - estrutura lógica de armazenamento, que pode ser composto de
diversas partições.
\end{itemize}

%%%%%%%%%%%%%%%%%%%%%%%%%%%%%%%%%%%%%%%%%%%%%%%%%%%%%%%%%%%%%%%%%%%%%%%%%%%%%%%
\section{Classificação}
%%%%%%%%%%%%%%%%%%%%%%%%%%%%%%%%%%%%%%%%%%%%%%%%%%%%%%%%%%%%%%%%%%%%%%%%%%%%%%%

Considera-se que a memória principal tem capacidade para armazenar
$M$ registros do arquivo a classificar.

%%%%%%%%%%%%%%%%%%%%%%%%%%%%%%%%%%%%%%%%%%%%%%%%%%%%%%%%%%%%%%%%%%%%%%%%%%%%%%%
\subsection{Classificação interna}

A {\bf classificação interna} envolve os seguintes passos:
\begin{itemize}
\item leitura de $M$ registros para a memória.
\item ordenação desses registros por um algoritmo de ordenação interna.
	\begin{itemize}
	\item quicksort, mergesort, etc.
	\end{itemize}
\item gravação desses registros classificados.
	\begin{itemize}
	\item em uma partição nova de um arquivo existente (ou em novo arquivo).
	\end{itemize}
\end{itemize}

A classificação interna possui a desvantagem de não explorar a ordem
parcial existente no arquivo de entrada.

%%%%%%%%%%%%%%%%%%%%%%%%%%%%%%%%%%%%%%%%%%%%%%%%%%%%%%%%%%%%%%%%%%%%%%%%%%%%%%%
\subsection{Seleção por substituição}

A {\bf seleção por substituição}  aproveita a possível classificação
parcial do arquivo de entrada. 
Em média, o tamanho das partições obtidas pelo processo de seleção com substituição
é de $2 * M$.

Os passos do algoritmo são:
\begin{enumerate}
\item Ler $M$ registros do arquivo para a memória.
\item Selecionar, na memória, o registro com menor chave.
\item Gravar na partição de saída o registro com menor chave.
\item Substituir (em memória) o registro gravado pelo próximo registro
do arquivo de entrada.
\item Caso a chave deste último seja menor do que a chave recém-gravada.
	\begin{enumerate}
	\item considerá-lo ``congelado'' e ignorá-lo no restante do processamento.
	\end{enumerate}
\item Caso existam em memória registros não ``congelados''
	\begin{enumerate}
	\item volta ao passo {\bf (2)}.
	\end{enumerate}
\item Caso contrário
	\begin{enumerate}
	\item fechar a partição de saída.
	\item ``descongelar'' os registros ``congelados''.
	\item abrir nova partição de saída.
	\item voltar ao passo {\bf (2)}.
	\end{enumerate}
\end{enumerate}

A seleção por substituição tem a desvantagem que no final da partição grande, 
parte do espaço em memória principal está ocupado com registros ``congelados''.

%%%%%%%%%%%%%%%%%%%%%%%%%%%%%%%%%%%%%%%%%%%%%%%%%%%%%%%%%%%%%%%%%%%%%%%%%%%%%%%
\subsection{Seleção natural}

A classificação por {\bf seleção natural} reserva um espaço na memória 
secundário (``\emph{o reservatório}''), que abriga os registros
``congelados'' em um processo de substituição.

A formação de uma partição se encerra quando o reservatório estiver cheio, ou
quando terminarem os registros de entrada.
Quando um reservatório encher, a memória é descarregada na partição atual
e os dados são copiados do repositório à memória.

A classificação por seleção natural usa como memória interna $M$ e um reservatório
que comporta $R$ registros.

%%%%%%%%%%%%%%%%%%%%%%%%%%%%%%%%%%%%%%%%%%%%%%%%%%%%%%%%%%%%%%%%%%%%%%%%%%%%%%%
\subsection{Comparação de processos}

A classificação interna gera as menores partições, mas simplifica o estágio
de intercalação por usar partições de mesmo tamanho.
Os processos de seleção exigem intercalação mais elaborada, porém geram 
partições maiores, e reduzem o tempo de processamento.
A seleçõa natural gera as maiores partições, porém utiliza mais operações de 
entrada e saída. Além de usar memória externa adicional.

Os processos de seleção podem utilizar Heapsort para a ordenação interna.

%%%%%%%%%%%%%%%%%%%%%%%%%%%%%%%%%%%%%%%%%%%%%%%%%%%%%%%%%%%%%%%%%%%%%%%%%%%%%%%
\section{Intercalação}
%%%%%%%%%%%%%%%%%%%%%%%%%%%%%%%%%%%%%%%%%%%%%%%%%%%%%%%%%%%%%%%%%%%%%%%%%%%%%%%

%%%%%%%%%%%%%%%%%%%%%%%%%%%%%%%%%%%%%%%%%%%%%%%%%%%%%%%%%%%%%%%%%%%%%%%%%%%%%%%
\section{Exercícios}
%%%%%%%%%%%%%%%%%%%%%%%%%%%%%%%%%%%%%%%%%%%%%%%%%%%%%%%%%%%%%%%%%%%%%%%%%%%%%%%

\begin{enumerate}
\item Dados os números $10, 8, 7, 11, 9, 13, 16, 12, 15, 14$, mostre as partições
que são criadas por classificação para os métodos: classificação interna, seleção
por substituição, seleção natural.

\end{enumerate}
