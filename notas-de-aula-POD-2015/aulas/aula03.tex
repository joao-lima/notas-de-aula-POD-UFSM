%%%%%%%%%%%%%%%%%%%%%%%%%%%%%%%%%%%%%%%%%%%%%%%%%%%%%%%%%%%%%%%%%%%%%%%%%%%%%%%
\chapter{Classificação de dados}
%%%%%%%%%%%%%%%%%%%%%%%%%%%%%%%%%%%%%%%%%%%%%%%%%%%%%%%%%%%%%%%%%%%%%%%%%%%%%%%

Os algoritmos deste capítulo resolvem o {\bf problema de ordenação}:
\begin{itemize}
\item {\bf Entrada}: uma sequência de $n$ números $\langle a_1, a_2, ..., a_n \rangle$.

\item {\bf Saída}:  Uma permutação $\langle {a'}_1, {a'}_2, ..., {a'}_n \rangle$ da
entrada tal que ${a'}_1 \leq {a'}_2 \leq ... \leq {a'}_n$.
\end{itemize}
Ordenação pode ser usado em diversos outros algoritmos.
Ela pode ser necessária devido a requisitos do usuário, ou para a otimização de pesquisa 
como na pesquisa binária.

Em geral os dados são mantidos em um vetor onde cada objeto possui um
atributo \textbf{chave} que deve ser mantido ordenado.
Para fins de exemplo, utiliza-se números inteiros como elementos.

Um algoritmo de ordenação possui duas características principais:
\begin{itemize}
\item {\bf Estabilidade} -- relativo a manutenção da ordem original dos itens 
com chaves iguais.
	\begin{itemize}
	\item Um algoritmo de ordenação é {\bf estável} se a ordem relativa dos itens
		com chaves iguais não se altera durante a ordenação.
	\end{itemize}
\item {\bf Uso de memória} -- quanto ao uso de memória pelo algoritmo.
	\begin{itemize}
	\item {\bf Com cópia de dados} -- utiliza um vetor temporário para realizar a ordenação. As trocas são feitas entre o vetor original e o temporário.
	\item {\bf In-place} -- as trocas são feitas dentro do próprio vetor original.
	\end{itemize}
\end{itemize}

Os métodos de ordenação podem ser {\bf interno} (em memória primária) ou {\bf externo} (em memória secundária).
Na {\bf interna} o arquivo de entrada cabe todo na memória principal, enquanto
que na {\bf externa} o arquivo não cabe na memória principal. 

A maioria dos métodos de ordenação é baseada em {\bf comparações} de chaves.
Porém, existem outros métodos que utilizam o principio da {\bf distribuição}. 
Um exemplo é ordenar um baralho com 52 cartas na ordem numérica e ordem de naipes.
O algoritmo seria:
\begin{enumerate}
\item Distribuir cartas em treze montes: ases, dois, três, ...., reis.
\item Coletar os montes na ordem especificada.
\item Distribuir novamente as cartas em quatro montes: paus, ouros, copas e espadas.
\item Coletar os montes na ordem especificada.
\end{enumerate}
Alguns desses métodos são o {\bf radixsort} e o {\bf bucketsort}.

%%%%%%%%%%%%%%%%%%%%%%%%%%%%%%%%%%%%%%%%%%%%%%%%%%%%%%%%%%%%%%%%%%%%%%%%%%%%%%%
\section{Classificação em memória primária}
%%%%%%%%%%%%%%%%%%%%%%%%%%%%%%%%%%%%%%%%%%%%%%%%%%%%%%%%%%%%%%%%%%%%%%%%%%%%%%%



%%%%%%%%%%%%%%%%%%%%%%%%%%%%%%%%%%%%%%%%%%%%%%%%%%%%%%%%%%%%%%%%%%%%%%%%%%%%%%%
\subsection{Bolha (\emph{bubble sort})}

%%%%%%%%%%%%%%%%%%%%%%%%%%%%%%%%%%%%%%%%%%%%%%%%%%%%%%%%%%%%%%%%%%%%%%%%%%%%%%%
\subsection{Seleção (\emph{selection sort})}

%%%%%%%%%%%%%%%%%%%%%%%%%%%%%%%%%%%%%%%%%%%%%%%%%%%%%%%%%%%%%%%%%%%%%%%%%%%%%%%
\subsection{Inserção (\emph{insertion sort})}

%%%%%%%%%%%%%%%%%%%%%%%%%%%%%%%%%%%%%%%%%%%%%%%%%%%%%%%%%%%%%%%%%%%%%%%%%%%%%%%
\subsection{Shellsort}

%%%%%%%%%%%%%%%%%%%%%%%%%%%%%%%%%%%%%%%%%%%%%%%%%%%%%%%%%%%%%%%%%%%%%%%%%%%%%%%
\subsection{Quicksort}

%%%%%%%%%%%%%%%%%%%%%%%%%%%%%%%%%%%%%%%%%%%%%%%%%%%%%%%%%%%%%%%%%%%%%%%%%%%%%%%
\subsection{Mergesort}

%%%%%%%%%%%%%%%%%%%%%%%%%%%%%%%%%%%%%%%%%%%%%%%%%%%%%%%%%%%%%%%%%%%%%%%%%%%%%%%
\subsection{Heapsort}

%%%%%%%%%%%%%%%%%%%%%%%%%%%%%%%%%%%%%%%%%%%%%%%%%%%%%%%%%%%%%%%%%%%%%%%%%%%%%%%
\subsection{Counting sort}

%%%%%%%%%%%%%%%%%%%%%%%%%%%%%%%%%%%%%%%%%%%%%%%%%%%%%%%%%%%%%%%%%%%%%%%%%%%%%%%
\subsection{Bucket sort}

%%%%%%%%%%%%%%%%%%%%%%%%%%%%%%%%%%%%%%%%%%%%%%%%%%%%%%%%%%%%%%%%%%%%%%%%%%%%%%%
\subsection{Radix sort}

%%%%%%%%%%%%%%%%%%%%%%%%%%%%%%%%%%%%%%%%%%%%%%%%%%%%%%%%%%%%%%%%%%%%%%%%%%%%%%%
\section{Classificação em memória secundária}
%%%%%%%%%%%%%%%%%%%%%%%%%%%%%%%%%%%%%%%%%%%%%%%%%%%%%%%%%%%%%%%%%%%%%%%%%%%%%%%

%%%%%%%%%%%%%%%%%%%%%%%%%%%%%%%%%%%%%%%%%%%%%%%%%%%%%%%%%%%%%%%%%%%%%%%%%%%%%%%
\section{Exercícios}
%%%%%%%%%%%%%%%%%%%%%%%%%%%%%%%%%%%%%%%%%%%%%%%%%%%%%%%%%%%%%%%%%%%%%%%%%%%%%%%

